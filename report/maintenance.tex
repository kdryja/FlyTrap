\chapter{Maintenance Manual}
% This appendix will provide information on how to compile and interpret source code of this project.
\section{Prerequisites}
To bring the whole framework up, three elements will be needed:
\begin{itemize}
  \item Linux - at the moment, the software stack is only compatible with Linux environments. It has been tested on ArchLinux and Ubutnu 16.04 - and for those distributions operation can be guaranteed. For other Linux distros, your mileage may vary.
  \item Golang - to compile (or run) all source files. All third-party Golang packages required by this project are managed through Go Modules\footnote{https://blog.golang.org/using-go-modules}, so there is no further action needed. Upon compilation, Go runtime will download all requirements and place them in your \$PATH (provided you have successfully installed Go). If desired, they can be manually inspected in `go.mod' file in project's root.
  \item Blockchain Node - either a dummy one provided by Ganache or a live node started through Geth. Remote nodes are also accepted.
  \item MQTT Broker - either a local broker (e.g. mosquitto on Ubuntu) or an online broker, reachable via TCP/TLS (e.g. test.mosquitto.org)
\end{itemize}
\section{Compiling Source Code}
Project consists of 4 binaries, which can be compiled as follows. Each `go build foo.bar' operation will produce a binary named `bar', which then can be executed by running \$ ./bar. To specify exact name of the produced binary, an extra -o flag can be provided. Note, compilation is optional and all sourcecode can be also interpreted by the runtime, this can be achieved by replacing `go build' with `go run'.

See below on how to compile binaries for each of the modules:
\begin{lstlisting}[language=bash,breaklines=true]
# compile blockchain CLI 
$ go build -o blockchain blockchain/cmd/blockchain/main.go 
# compile flytrap
$ go build -o flytrap flytrap/cmd/flytrap/main.go
# compile mqttclient 
$ go build -o mqttclient mqttclient/client.go
# compile webapp. note, 'templates' folder has to be adjacent to the produced binary when it is executed 
$ go build -o webapp webapp/main.go 
\end{lstlisting}
\section{Project File Tree}
\begin{forest}
  for tree={font=\sffamily, grow'=0,
  folder indent=.9em, folder icons,
  edge=densely dotted}
  [root
    [blockchain
      [cmd
        [blockchain
          [main.go, is file]]]
      [blockchain.go, is file]
      [flytrap.go, is file]
      [flytrap.sol, is file]
      [logging.go, is file]
      ]
    [flytrap
      [cmd
        [flytrap
          [main.go, is file]]]
      [proxy.go, is file]
      [summary.go, is file]
      [verification.go, is file]
      [GeoLite2-Country.mmdb, is file]
    ]
    [mqttclient
      [client.go, is file]]
    [webapp
      [templates
        [index.html, is file]
        [script.js, is file]]
      [main.go, is file]]
    [evaluation]
  ]
\end{forest}
\section{Files Overview}
Files that can be compiled and are considered main entry point for the given component are marked with an asterisk (\textbf{*})
\subsection{Blockchain}
This module contains code responsible for communicating with blockchain.
\paragraph{blockchain.go} contains all logic related to making calls to Blockchain to retrieve or write data. It's referenced by FlyTrap when checking access permissions.
\paragraph{cmd/blockchain/main.go *} main executable and CLI when communicating with blockchain. Executing this file. 
\paragraph{flytrap.sol} smart contract definition written in Solidity.
\paragraph{flytrap.go} Solidity code translated into Golang, which then can be used to make calls towards blockchain.
\paragraph{logging.go} logic to query state of blockchain in order to pull logs to be used in web frontend.
\subsection{Flytrap}
This module contains the main logic behind FlyTrap proxy. 
\paragraph{cmd/flytrap/main.go *} main executable of FlyTrap proxy.
\paragraph{summary.go} code containing logic to cache publishers/subscribers and dump them periodically as a transaction to blockchain.
\paragraph{verification.go} logic to query the GeoLite2 database in order to translate IP into a country that it is located in.
\paragraph{GeoLite2-Country.mmdb} MaxMind database file used to lookup IP addresses.
\subsection{MQTT Client}
\paragraph{client.go *} simple MQTT client used to PUBLISH/SUBSCRIBE to MQTT Brokers, enhanced with extra functionality needed by FlyTrap calculating and attaching signature to CONNECT packets.
\subsection{Webapp}
This module is used to start up a frontend instance for FlyTrap allowing inspecting of data stored on blockchain.
\paragraph{templates} HTML + JS files used to render the website.
\paragraph{main.go *} backend used to handle incoming requests and query blockchain, returning information filling the website.
\subsection{Evaluation}
This folder contains dump of all tests results performed for the purposes of evaluation.
