\chapter{Requirements}
In this chapter, we outline the base requirements for the project along with sample stories that would later dictate the workflow.

\section{Requirements}

The project can be divided into a set of both functional and non-functional requirements (FR and NFR). Those provide success criteria for the  produced software.

\subsection{Functional Requirements}
Set of requirements defining desired functionality of the system.

\begin{itemize}[leftmargin=4.5em]
\item[\textbf{(FR1)}] The system will provide an interface to manage access to the topics along with inspecting the audit trails.
\item[\textbf{(FR2)}] The system can connect to any Ethereum node, be it a public endpoint or a locally running, closed network. This will provide the flexibility of either using transparent and with 100\% uptime resource or a closed node with reduced costs.
\item[\textbf{(FR3)}] The system should provide a way to collect payments in ETH from clients attempting to gain access to relevant resources. This payment would then in the process, be transferred to the resource owner's Ethereum wallet.
\item[\textbf{(FR4)}] The system should offer an option to specify an exact amount of ETH required to publish or subscribe - with the possibility of separating the costs and also setting the cost to 0 (free).
\item[\textbf{(FR5)}] The system should be capable of fending off primitive denial-of-service attacks by blocking continuous, failed attempts to connect.
\item[\textbf{(FR6)}] The operations performed by clients will be of limited complexity, such that they can be executed on devices with limited computational power.
\item[\textbf{(FR7)}] The system can answer crucial GDPR questions, such as who accessed a given resource, why did they have access, when they accessed it and what exactly was accessed.
\item[\textbf{(FR8)}] The system should offer an option to restrict the client's country that can access the resource, which will be verified using GeoIP lookup, as various countries have various data protection laws.
\end{itemize}

\subsection{Non-functional Requirements}

In addition to the functional, it is also vital to mention the following non-functional requirements, as the system is intended for end-users (potentially non-technical) and due to incorporation with blockchain can introduce performance overhead.

\begin{itemize}[leftmargin=4.5em]
\item[\textbf{(NFR1)}] The system should provide \textbf{an overhead of no more than half a second} per MQTT session. This is important, as the intention is to provide an add-on on top of the existing MQTT brokers. This might further compromise the current efficiency, so the system should aim to minimise the added latency. 
\item[\textbf{(NFR2)}] The system should be agnostic of the used MQTT broker, as long as the broker \textbf{fully implements the MQTT v5.0 standard}. As pointed out earlier, there is a variety of brokers available to use, such as Mosquitto or Moquette. FlyTrap should not rely on the implementation of a broker, but rather only on the standard utilised.
\item[\textbf{(NFR3)}] The system should be capable of extending any MQTT broker with \textbf{Authentication, Authorisation, Accountability} framework. This is to ensure that data can only be accessed by authenticated entities, which are authorised to access requested resources - and in case of a breach or other disaster, keep them accountable to their actions.
\item[\textbf{(NFR4)}] The system should only be based on \textbf{Open-Source Software}. Since the ultimate aim is to provide increased security, keeping the source open would allow any potential users to inspect its operation. Furthermore, third party security audits can happen without the system owner's intervention.
\end{itemize}


\subsection{User stories}
User stories are further aiming to formulate the requirements into more concise sample situations:
\begin{enumerate}
\item As a government regulator, I would like to overview access history to specific MQTT topics, to make sure the data is handled in a GDPR compliant manner.
\item As a government regulator, I would like to verify why / when / who accessed a given resource at a specific time, such that I can issue fines for potential non-compliance and inspect data breaches.
\item As a topic owner, I would like to restrict people that can publish / subscribe to them, to maintain their confidentiality.
\item As a topic owner, I would like to collect payments from people willing to access my data.
\item As a topic owner, I would like to block access to my information from requests coming from outside the requested country, to comply with GDPR requirements.
\item As a broker owner, I would like to collect payments from people willing to publish their data on my system, to keep the system profitable.
\item As a broker owner, I would like to secure a distributed network of brokers (with varied implementations), to increase the system's availability. 
\item As a broker owner, I would like to block access to the system to malicious clients performing denial-of-service attacks, to avoid system downtime.
\item As a data consumer, I would like to publish/subscribe my messages on low-power devices, such that I can utilise my IoT sensors.
\item As a data consumer, I would like to access the broker from over a hundred parallel sensors, each publishing data independently.
\end{enumerate}

\subsection{Use-case Scenarios}\label{sec:usecase}
Below I have listed a collection of stories which relate to hypothetical scenarios and present various problems faced in different areas - both in industrial and commercial environments. FlyTrap is looking to address the issues surfaced in those short stories. In the Evaluation chapter of this paper, they are referred to and tested on whether it is indeed helpful and how the discussed solution solves the problems.
\\
\subsubsection{Scenario \#1: Air Quality study in the UK}
\paragraph{\textbf{Scenario:}}
Robert is working as a Research Fellow at a University located in Manchester. His research aims to issue air quality IoT sensors to staff across University, intending to capture information such as pollution or carbon dioxide level to analyse the contents of air in the state. Each sensor is issued to an individual taking part in an experiment (e.g. member of staff, lecturer, PhD student), which is based in a specific room on campus. Robert needs to be able to track the inventory, and thus every sensor must be traceable down to a person.

The budget allows to issue up to 1000 sensors, and Robert would like to use the MQTT broker to receive the data from the IoT devices. Additionally, he would like to share the dataset with researchers across the country; thus, he makes the MQTT broker public. As per GDPR, such data, containing full name, office location and detailed temperature readings, is fully protected and needs to adhere to various governance requirements within the European Union. This information should also be stored only within the jurisdiction, that is, only within EU, to ensure enforceability of GDPR. Robert needs to make sure that only researchers that are located in Europe can access the data and everyone else (for example, people from the USA or Russia) will be denied access.

\paragraph{\textbf{What problem is addressed here:}}
Data containing personal information of European citizens needs to be handled in a GDPR compliant way. This cannot be ensured if the information is accessible outside EU, as there would be no jurisdiction or enforceability once the data leaves the region. The person responsible for data storage needs to ensure that no people outside EU can access it.
\paragraph{\textbf{Relevant requirements:}} FR1, FR6, FR7, FR8.
\\
\subsubsection{Scenario \#2: Data breach in an oil drilling facility}
\paragraph{\textbf{Scenario:}}
Bob is a Chief Information Technology in a company Chell handling processing of oil and gas in Scotland. Bob's company also contracts many smaller companies which provide staffing and direct drilling services. Many sensors are used in the company, which are responsible for collecting data such as air pressure, humidity, occupancy on drilling platform or temperature. Those IoT sensors are utilising an MQTT Broker, which is restricted only to authorised Chell employees.

Unfortunately, due to unrelated reasons, access to the broker has been compromised thus allowing third parties to peek into the data flowing through potentially. Bob is approached by Judy, who works with the team responsible to trace the severity of leakage and determine the potential scope \& damages. Judy asks Bob to outline who might have had access to the leaked information and what the leaked information contained. Judy also instructs Bob to inform all people and contractors that might have been affected by the breach.
\paragraph{\textbf{What problem is addressed here:}}
Companies that use MQTT brokers to store information which is confidential and contains trade secrets of high commercial value want to determine the range of leak that happened. This is part of disaster recovery procedure and is important for business continuity; i.e. determine how much of data has leaked and what kind of information was exposed to malicious actors.
\paragraph{\textbf{Relevant requirements:}} FR1, FR7.
\\
\subsubsection{Scenario \#3: Unsatisfied Customer}
\paragraph{\textbf{Scenario:}}
Mary recently purchased a smart assistant, which comes with several smart sensors to be placed around the house. Unfortunately, Mary decided to return all the sensors and cease further usage, she reaches out to Moogle's representative - Matt.

Matt knows that Moogle is using MQTT brokers to connect their smart sensors and then use the phone app to issue commands back to them through the broker. Although the phone app is not the only piece of software that has access to the data from the smart sensors. Analytics teams also consume those in order to help Moogle create better products. Those servers have their own data storage capabilities and replicate the data that they consume. Matt is now tasked with identifying which internal analysis services might have accessed Mary's sensors in order to erase this information since it is one of the GDPR requirements, also called "right to be forgotten".
\paragraph{\textbf{What problem is addressed here:}}
Again, GDPR comes into action here, in particular Article 17 - Right to erasure. Moogle needs to permanently erase all trail coming from Mary's sensors, that includes any analytics datasets. Since those services are using MQTT brokers, there is no access trail and without proper infrastructure, it is impossible to go in the past and track which services were accessing the data.
\paragraph{\textbf{Relevant requirements:}} FR1, FR7.
\\
\subsubsection{Scenario \#4: Monetization of data}
\paragraph{\textbf{Scenario:}}
Frank owns several farm fields in Scottish Highlands. As part of the ongoing digitalisation, he decided to purchase some IoT sensors to learn more about the environment his farms are located in. Some of the sensors include specialised soil acidity meters capable of measuring pH or determine how moist it is.

He decided to connect all sensors to one MQTT broker for ease of access and started to consume the information through the provided mobile application. As the equipment was quite expensive, Frank would like to also sell access to the sensors to the interested parties, such as other land owners or University researchers looking to learn more about the soil quality in the area. No personal information is included, though the data is highly dependant on the location (which currently only Frank can access) and thus might be also useful for other people. 
\paragraph{\textbf{What problem is addressed here:}}
Selling data is a very lucrative business, especially if the data is valuable and has potential audience. MQTT does not have provisions to accommodate such requirements - anything like that would need to either be stored in some persistent location (and then processed for commercial purposes) or routed through an extra layer of verification.
\paragraph{\textbf{Relevant requirements:}} FR1, FR3, FR4.
\\
\subsubsection{Scenario \#5: Securing access to the broker}
Bob works for a start-up company contracted by the local council to deploy smart sensors across the city to measure various statistics, such as air quality, traffic intensity, parking spot occupancy. All of those sensors are connected to the MQTT broker, which later can also be accessed by third parties - providing they have a legitimate and approved reason. The MQTT broker is accessible by people that might not have overlapping permissions, for example company A can access only data set Y and company B can access only data set X - both X \& Y are published on the same MQTT broker.

Bob is aware of the vanilla implementation of username/password authentication for MQTT brokers, but is concerned about the insufficient security in case of leakage. He also already is using Ethereum within the company to hold internal data, so can issue more wallets - if necessary.
\paragraph{\textbf{What problem is addressed here:}}
Security capabilities of vanilla MQTT Brokers can be found lackluster and not sufficient for many needs. Using symmetric password authentication can be very damaging if the said password leaks to third parties. For people that are already using related blockchain technologies, the transition to FlyTrap would be even simpler, as it can make use of any and every wallet compatible with ETH.
\paragraph{\textbf{Relevant requirements:}} FR1, FR2, FR5.
