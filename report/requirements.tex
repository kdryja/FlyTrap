\chapter{Requirements \& Architecture}
In this chapter, I will outline base requirements for the project along with sample stories that would later dictate the workflow. In the second part, I will also include an overview of the architecture proposed for the system, correlating each element with relevant requirement and explaining how they would address the use-cases.

\section{Requirements}

\subsection{User stories}

\begin{enumerate}
\item As a government regulator, I would like to overview access history to specific MQTT topics, to make sure the data is handled in GDPR-compliant manner.
\item As a government regulator, I would like to verify why / when / who accessed given resource at a specific time, such that I can issue fines for potential non-compliance and inspect data breaches.
\item As a topic owner, I would like to restrict people that can publish/subscribe to them, to maintain their confidentiality.
\item As a topic owner, I would like to collect payments from people willing to access my data.
\item As a topic owner, I would like to block access to my information from requests coming outside the requested country, to comply with GDPR requirements.
\item As a broker owner, I would like to collect payments from people willing to publish their data on my system, to keep the system profitable.
\item As a broker owner, I would like to secure a distributed network of brokers (with varied implementations), to increase the system's availability. 
\item As a broker owner, I would like to block access to the system to malicious clients performing denial-of-service attacks, to avoid system downtime.
\item As a data consumer, I would like to publish/subscribe my messages on low-power devices, such that I can utilise my IoT sensors.
\item As a data consumer, I would like to access the broker from over a hundred parallel sensors, each publishing data independently.
\end{enumerate}

\subsection{Use-case Scenarios}
A collection of real-world problems that this project is trying to address.
\\
\subsubsection{Scenario \#1: Air Quality study in Texas}
\paragraph{\textbf{Scenario:}}
Robert is working as a Research Fellow at a University located in Texas. The research aims at issuing air quality IoT sensors to staff across University, intending to capture information such as pollution or carbon dioxide level to analyse contents of air in the state. Each sensor is issued to an individual taking part in an experiment (e.g. member of staff, lecturer, PhD student), which is based in a specific room on campus. Robert needs to be able to track the inventory, and thus every sensor must be trackable down a person.

The budget allows to issue up to 1000 sensors, and Robert would like to use MQTT broker to receive the data from the IoT devices. Additionally, he would like to share the dataset with researches across the country; thus, he makes the MQTT broker public. As per GDPR, such data, containing full name, office location and detailed temperature readings are fully protected and needs to adhere to various governance requirements within the European Union. Unfortunately, Robert does not have resources nor funding to ensure that the data is kept and flowing in GDPR compliant manner; thus he wants to restrict access to the broker only to people connecting from the US.
\paragraph{\textbf{What problem is addressed here:}}
Some companies do not have enough resources to ensure compliance with GDPR. However, it does not relieve them from the necessity of compliance, if the data is accessible from the EU and if the data contains personal data of an EU citizen. So some people decide to block access from the EU altogether. 
\\
\subsubsection{Scenario \#2: Data breach in an oil drilling facility}
\paragraph{\textbf{Scenario:}}
Bob is a Chief Information Technology in a company Chell handling processing of oil and gas in Scotland. Bob's company also contracts many smaller companies which provide staffing and direct drilling services. Many sensors are used in the company, which are responsible for collecting data such as air pressure, humidity, occupancy on drilling platform or temperature. Those IoT sensors are utilising MQTT Broker, which is restricted only to authorised Chell employees.

Unfortunately, due to unrelated reasons, access to the broker has been compromised and thus allowing third parties to peek into the data flowing through potentially. Bob is approached by Judy, who works with the UK Government and is concerned about the leakage. Judy asks Bob to outline who might have had access to the leaked information and what the leaked information contained, as described by Article 33 and 34 of GDPR. Judy also instructs Bob to inform all people and contractors that might have been affected by the breach
\paragraph{\textbf{What problem is addressed here:}}
Compliance with Art. 33 \& 34 of GDPR. Vanilla MQTT has no logging on who might have access to the information, nor what information was accessed at specific timelines.
\\
\subsubsection{Scenario \#3: Unsatisfied Customer}
\paragraph{\textbf{Scenario:}}
Mary recently purchased a smart assistant, which comes with several smart sensors to be placed around the house. Those sensors consist of devices such as a smart doorbell, smart thermostat, smart kettle or even smart window blinds - all produced and managed by a company called Moogle. Mary can use her mobile phone to change the temperature at her home or pull up the blinds remotely. Unfortunately, due to the poor sensor quality and concerns about Moogle's management of personal information, Mary decided to return all the sensors and cease further usage, she reaches out to Moogle's representative - Matt.

Matt knows that Moogle is using MQTT brokers to connect their smart sensors and then use the phone app to issue commands back to them through the broker. Although the phone app is not the only piece of software that has access to the data from the smart sensors. Analytics teams also consume those in order to help Moogle create better products. Matt is now tasked with identifying which internal analysis services might have accessed Mary's sensors in order to erase this information since it is a GDPR requirement, also called "right to be forgotten".
\paragraph{\textbf{What problem is addressed here:}}
Again, GDPR comes into action here, in particular Article 17 - Right to erasure. Moogle needs to permanently erase all trail coming from Mary's sensors, that includes any analytics datasets. Since those services are using MQTT brokers, there is no access trail and without proper infrastructure, impossible to go in the past and track which services were accessing the data

\subsection{Functional Requirements}

The user stories can then be further formulated into the following functional requirements:

\begin{itemize}[leftmargin=4.5em]
\item[\textbf{(FR1)}] The system will provide an interface to manage access to the topics along with inspecting the audit trails.
\item[\textbf{(FR2)}] The system can connect to any Ethereum node, be it a public endpoint or a locally running, closed network. This will provide the flexibility of either using transparent and with 100\% uptime resource or a closed node with reduced costs.
\item[\textbf{(FR3)}] The system should provide a way to collect payments in ETH from clients attempting to gain access to relevant resources. This payment would then in the process, be transferred to the resource owner's Ethereum wallet.
\item[\textbf{(FR4)}] The system should offer an option to specify an exact amount of ETH required to publish or subscribe - with the possibility of separating the costs and also setting the cost to 0 (=free).
\item[\textbf{(FR5)}] The system should be capable of fending of primitive denial-of-service attacks by blocking continuous, failed attempts to connect.
\item[\textbf{(FR6)}] The operations performed by clients will be of limited complexity, such that they can be executed on devices with limited computational power.
\item[\textbf{(FR7)}] The system can answer crucial GDPR questions, such as who accessed given resource, why did they have access, when they accessed it and what exactly was accessed.
\item[\textbf{(FR8)}] The system should offer an option to restrict the client's country that can access the resource, which will be verified using GeoIP lookup, as various countries have various data protection laws.
\end{itemize}

\subsection{Non-functional Requirements}

In addition to the functional, it is also vital to mention the following non-functional requirements, as the system is intended for end-users (potentially non-technical) and due to incorporation with blockchain can introduce performance overhead.

\begin{itemize}[leftmargin=4.5em]
\item[\textbf{(NFR1)}] The system should provide \textbf{an overhead of no more than 2 seconds cumulative} per MQTT session. This is important, as the intention is to provide an add-on on top of the existing MQTT brokers. This might further compromise the current efficiency, so the system should aim to minimise the added latency 
\item[\textbf{(NFR2)}] The system should be agnostic of the used MQTT broker, as long as the broker \textbf{fully implements MQTT v5.0 standard}. As pointed out earlier, there is a variety of brokers available to use, such as Mosquitto or Moquette. FlyTrap should not rely on the implementation of a broker, but rather only on the standard utilised.
\item[\textbf{(NFR3)}] The system should be capable of extending any MQTT broker with \textbf{Authentication, Authorisation, Accountability} framework. This is to ensure that data can only be accessed by authenticated entities, which are authorised to access requested resources - and in case of a breach or other disaster, keep them accountable to their actions.
\item[\textbf{(NFR4)}] The system should only be based on \textbf{Free and Open-Source Software}. Since the ultimate aim is to provide increase security, keeping the source open would allow any potential users to inspect its operation. Furthermore, third party security audits can happen without the system owner's intervention.
\item[\textbf{(NFR5)}] The system should be capable to run inside \textbf{virtualised container}, to ensure that it's platform agnostic.
\end{itemize}

\section{Architecture}
